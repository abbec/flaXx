\documentclass[a4paper]{report}

\usepackage{graphicx}
\usepackage{amsmath, amsthm}

% To get slightly smaller margins
\usepackage{fullpage}

% To get paragraphs right
\usepackage[parfill]{parskip}

% Correct input encoding
\usepackage[utf8x]{inputenc}

\begin{document}

\title{FlaXx: A Monte-Carlo (Stochastic) Ray Tracer}
\author{Nathalie Ek \and Albert Cervin}

\maketitle

\begin{abstract}
Abstract goes here.
\end{abstract}

\tableofcontents

\listoffigures

\chapter{Introduction}

\section{Global Illumination}

The problem of global illumination has existed just as long as images
have been produced with computers. The problem consists of solving the
rendering equation. The hemispherical formulation of the rendering equation is

\begin{equation}
  L(x \to \Theta) = L_e(x \to \Theta) + \int_{\Omega_x}f_r(x,\Psi \to \Theta)L(x \gets \Psi)\cos{N_x,\Psi}d\omega_\Psi
  \label{eq:renderingeq}
\end{equation}

where \(L(x \to \Theta)\) is the radiance going from the point \(x\)
in direction \(\Theta\), \(L_e(x \to \Theta) \) is the self-emitted
radiance from the point \(x\) in direction \(\Theta\), \(f_r(x,\Psi
\to \Theta)\) is the Bidirectional Reflectance Distribution Function
(BRDF) which tells how much of the incoming light from direction
\(\Psi\) to the point \(x\) is leaving in direction \(\Theta\). The
BRDF is material-dependent and will look different for different materials.

The first usable solution to this problem was proposed by Turner
Whitted \cite{whitted} in 1980. Whitted proposed a ray-tracing scheme
where only perfect specular reflection and refraction is
considered. To take diffuse interactions into consideration, a local
lighting model is used where rays are cast towards the light source
from the ray intersection point with an object. This solves the
problem with shadows and diffuse objects. However, this method stops
the recursion as soon as a diffuse surface is hit and thus does not
take diffuse interreflections into consideration.

This first section is an introduction to the problem and different
solutions proposed. In the next chapter Monte Carlo raytracing will be
discussed more in detail, and details of our implementation will be
presented. In chapter \ref{ch:results} results from the implementation
is presented and discussed. Discussion and outlook is presented in chapter \ref{ch:discussion}. 

\chapter{Background}

\chapter{Results and benchmarks}
\label{ch:results}


\chapter{Discussion}
\label{ch:discussion}


% References
\bibliographystyle{plain}
\bibliography{refs}

\end{document}